\documentclass[compress,red]{beamer}
\usetheme{Warsaw}
\useoutertheme[subsection=false]{smoothbars}

\usepackage{graphicx}
\usepackage{booktabs}
\usepackage{hyperref}
\usepackage{url}

\usepackage{remreset}% tiny package containing just the \@removefromreset command
\makeatletter
\@removefromreset{subsection}{section}
\makeatother
\setcounter{subsection}{1}

\title[FOSS at CERN: Drivers in the Kernel]%
	{Free and Open Source Software at CERN:\\
	Integration of Drivers in The Linux Kernel}
\author[David Cobas et al.]{%
	Juan David Gonz\'alez Cobas, Samuel Iglesias Gons\'alvez,\\
	Julian Howard Lewis, Javier Serrano, Manohar Vanga\\
		(CERN, Geneva),\\
	Emilio G. Cota (Columbia University, NY; formerly at CERN),\\
	Alessandro Rubini, Federico Vaga (University of Pavia)}
\date{ICALEPCS'2011}

\begin{document}

\begin{frame}
\titlepage
\end{frame}

\section{The tsi148 bridge}
\begin{frame}{CERN Controls System Front End Computers (FECs)}

The controls system relies on FECs on several form factors/buses,
most of them based on Single-Board Computers (SBCs)

\begin{itemize}
\item Number of FECs: 1140
\item Number of VME crates: 710
\end{itemize}

For the VME crates, the ongoing renovation process gives
\begin{itemize}
\item CES RIO2/RIO3 SBCs with PowerPC CPUs runing
LynxOS (around 605 crates by August 2011), to
\item MEN-A20 SBCs with Intel CPUs running real-time
Linux (around 105 by August 2011).
\end{itemize}
\end{frame}


\begin{frame}{MEN A20 and the TSI148 chip and driver}

The MEN A20 SBC is an Intel Core 2 Duo-based board interfacing to the
VME bus via a Tundra TSI148 PCI-X to VME bridge chip.

% wget http://www.men.de/docs-ext/products/images/img_photo/01a020-_hi.jpg
% ftp://ftp.ruby-lang.org/pub/ruby/1.8/ruby-1.8.7-p352.tar.gz
\begin{center}
\includegraphics[height=4cm]{01a020-_hi.pdf} \qquad
\includegraphics[height=2.75cm]{p_tundra_Tsi148-HR.pdf}
\end{center}
\end{frame}

\begin{frame}{A driver for the \texttt{tsi148}}

Some data about our driver for the \texttt{tsi148} PCI-X to VME bridge.
\begin{itemize}
\pause\item Developed at CERN in spring 2009 (S\'ebastien Dugu\'e)
\pause\item Maintained and extended by Emilio G. Cota during 2010
\pause\item Currently maintained by Manohar Vanga (see below)
\pause\item API via exported symbols (kernel) and old-style \texttt{ioctl()} (user) interface.
\pause\item Backward compatible at the API level with the original LynxOS CES
    library (well, almost) and offering a newer API as well
\end{itemize}

\pause Beginning as an \textcolor{red}{in-house}
and \textcolor{red}{CERN-centric} development
\end{frame}

\begin{frame}{Why going upstream? (2010)}

By mid-2010, the decision is taken to submit the driver for acceptance
in the Linux kernel main tree. Motivation:

\begin{itemize}
\item Smoother maintenance in the (frequent) event of kernel API
  changes\\
  (see \texttt{Documentation/stable\_api\_nonsense.txt} in the kernel
  tree).
\item Widespread distribution of the code base, which can then be
    enhanced and get contributions by researchers working on similar
    problems.
\item Contributing back in return to the many benefits the FOSS community
    gives us.
\end{itemize}
The original motivations were \textcolor{red}{less practical than ideological}
\end{frame}

\begin{frame}{Stages in the process}

We underwent the merge with the pre-existing \texttt{./staging/}
driver for the Tundra Universe and TSI148 chips,
maintained by Martyn Welch (GE).

\begin{description}
\pause
\item[nov 2010] First submission of patches by Emilio G. Cota.
    Modification of the device model implementation. Partially accepted,
    bus/driver issues not accepted.
\pause
\item[march 2011] Second attempt to push patches by Manohar Vanga.
    Most stylistic modifications accepted. Device model modifications
    controversial. Scheduled a second round.
\pause
\item[mid 2011] Re-submission of the corrected patch set. Core
    modifications to the device model accepted.
\end{description}
\end{frame}

\begin{frame}{Lessons learned}
\begin{itemize}
\item It is hard, frustrating at times. One should be ready for the
    peculiar culture of the Linux Kernel Mailing List.
\item Maintainers are occasionally hard to deal with.
\item One must be prepared to compromise.
\item Design, APIs and development must be adapted to comply with the strict
    standards imposed by the Linux kernel developers.
\item The review process may be long, with several iterations.
\item Having a good history of prior contributions gives respectability
    in that meritocratic culture.
\item Small, incremental changes are more likely to be accepted than
    big, hard-to-digest ones.
\end{itemize}
\end{frame}

\begin{frame}{Lessons learned}
But the most important one was that our initial motivations
\pause
\textcolor{red}{turned out to be wrong}

\pause
\begin{center}
\Huge Why?
\end{center}
\end{frame}

\section{FMC boards}

\begin{frame}{The FMC family of boards}

This is a substantial part of our standard HW kit, currently under
development\\
(see \url{http://www.ohwr.org/projects/fmc-projects}).

\begin{itemize}
\item carriers in PCIe and VME format
\item simple mezzanines with electronics for ADCs, DACs, DIO and endless
    other applications
\item circuitry in the mezzanine
\item FPGA application logic in the carrier
\item \emph{logic in the FPGA is organized as a set of cores
    interconnected through a Wishbone bus}
\end{itemize}
\end{frame}

\begin{frame}{A typical data acquisition application: carrier}
\begin{center}
\includegraphics[height=0.8\textheight]{spec.pdf}
\end{center}
\end{frame}

\begin{frame}{A typical data acquisition application: mezzanine}
\begin{center}
\includegraphics[rotate=180,height=0.7\textheight]{fmcadc.pdf}
\end{center}
\end{frame}

\begin{frame}{What's inside the FPGA}
\includegraphics[height=0.8\textheight]{THCHMUST04f2_talk.pdf}
\end{frame}

\begin{frame}{Architecture of the FMC drivers}
\includegraphics[height=0.8\textheight]{THCHMUST04f1.pdf}
\end{frame}

\begin{frame}{Drivers for the FMC family}
The main concepts for the design of these drivers are
\begin{itemize}
\pause
\item modular structure that reflects the core structure of the firmware
\pause
\item one-to-one mapping driver $\leftrightarrow$ core (usually)
\pause
\item wishbone-centricity
\pause
\item ability to dynamically load bitstreams by application
\end{itemize}
\pause
On the whole, the driver for the carrier board acts as a basic firmware
loader and a Wishbone bus driver (with device enumeration
\emph{\`a la PCI}

\pause
It will be (we hope) the first Wishbone bus driver in the mainstream
kernel $\Rightarrow$ will to go upstream, timeliness
\end{frame}

\section{The \texttt{zio} framework}

\begin{frame}{BE/CO data acquisition modules}
   \begin{tiny}
   \begin{tabular}{llllll}
       \toprule
	\textbf{Module}& \textbf{Type}& \textbf{Channels}&
	\textbf{Resolution}& \textbf{Max. Speed}& \textbf{Bus} \\
       \midrule
	VMOD-12E8/16    &  Analog input  & 8/16ch & 12b    & 15us/sample & VME/PCI  \\
	VD80            &  Analog input  & 16ch   & 16b    & 200kS/s     & VME  \\
	SIS3300         &  Analog input  & 8ch    & 12/14b & 100MS/s     & VME  \\
	SIS3302         &  Analog input  & 8ch    & 16b    & 100MS/s     & VME  \\
	SIS3320         &  Analog input  & 8ch    & 12b    & 250MS/s     & VME  \\
	``Fast'' FMC ADC&  Analog input  & 4ch    & 14b    & 100Ms/s	 & VME/PCIe (WB)  \\
	``Slow'' FMC ADC&  Analog input  & 8ch    & 16b    & 100kS/s     & VME/PCIe (WB)  \\
       \midrule
	CVORB V4        &  Analog output & 16ch   &  16b   &  5us/sample & VME 	 \\
	VMOD-12A2/4     &  Analog output & 2ch    &  12b   &  10us/sample& VME/PCI \\
	CVORG           &  Analog output & 2ch    &  14b   &  100 MS/s   & VME  \\
       \midrule
	VMOD-TTL        &  Digital I/O   & 20ch   & 1b     & n/a         & VME/PCI \\
	CVORA           &  Digital I/O   & 32ch   & 1--32b & 100Mhz	 & VME \\
       \bottomrule
   \end{tabular}
   \end{tiny}
\end{frame}

\begin{frame}{Industrial I/O frameworks}

There are two main frameworks in the Linux (staging) kernel for
industrial data acquisition applications
\pause
\begin{itemize}
\item Comedi
\item IIO
\end{itemize}

\pause
Careful analysis shows that
\begin{itemize}
\pause
\item our requirements are not suited by them
\pause
\item the interfaces offered can be cumbersome
\end{itemize}

\pause
A generic, modern framework to tackle these heterogeneous device drivers
is under development by Alessandro Rubini and Federico Vaga, under the
codename~\texttt{zio}. It is also intended for integration in the
mainstream kernel. See
\url{git://gnudd.com/zio-beta.git}
\end{frame}

\begin{frame}{\texttt{zio} features}

The aspects that the \texttt{zio} framework intends to cover are
manifold
\begin{itemize} \small
\pause\item Digital and analog input and output.
\pause\item One-shot and streaming (buffered) data acquisition or waveform play.
\pause\item Resolution and sampling rate.
\pause\item Buffer management and timing for streaming conversion.
\pause\item Support for DMA.
\pause\item Calibration, offset and gain.
\pause\item Bit grouping in digital I/O.
\pause\item Timestamping.
\pause\item Triggering of acquisition/output.
\pause\item \textcolor{red}{Clean design conforming to Linux kernel practice}
\end{itemize}
\end{frame}

\begin{frame}{Next candidates}

CERN-developed drivers for
\begin{itemize}
\pause
\item Struck SIS33xx ADCs
\pause
\item Tews TPCI200/TVME200 carries plus IPOCTAL serial boards
\pause
\item all the CERN BE/CO-supported FMC boards in the Open Hardware Repository
\pause
\item timing receivers, White Rabbit, etc.
\end{itemize}
\end{frame}

\section{Conclusions}

\begin{frame}{Why did we do it?}

It gave us much more than we thought in the first place
\begin{itemize}
\pause
\item Smoother maintenance in the (frequent) event of kernel API
  changes.
\item Widespread distribution of the code base.
\item Contributing back to the the FOSS community.
\pause \color{red}
\item Very strict process of peer review of the code by knowledgeable
    and specialised maintainers.
\pause
\item Input (consulting!) from the topmost experts in the field.
\pause
\item Avoidance of suboptimal, \emph{ad hoc} solutions in favour of the
    best ones from the technical point of view.
\pause
\item Use of best practice and bleeding-edge tools selected by
    experienced programmers, \emph{e.g.} \texttt{git}, \texttt{sparse}
    and \texttt{coccinelle}.
\end{itemize}
\end{frame}

\begin{frame}{The ultimate goal}
    Being able to drive critical hardware components with software
    in the vanilla kernel, with no local, idiosyncratic modifications
    is the ultimate goal (or so we believe).

    Some steps are going in that direction now (e.g. CERN BE/CO
    supported kernel)
\end{frame}

\end{document}
